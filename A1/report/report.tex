\documentclass[12pt]{article}

\usepackage{graphicx}
\usepackage{paralist}
\usepackage{listings}
\usepackage{booktabs}
\usepackage{hyperref}

\oddsidemargin 0mm
\evensidemargin 0mm
\textwidth 160mm
\textheight 200mm

\pagestyle {plain}
\pagenumbering{arabic}

\newcounter{stepnum}

\title{Assignment 1 Solution}
\author{Hriday Jham, jhamh}
\date{\today}

\begin {document}

\maketitle

This report discusses testing of the \verb|ComplexT| and \verb|TriangleT|
classes written for Assignment 1. It also discusses testing of the partner's
version of the two classes. The design restrictions for the assignment
are critiqued and then various related discussion questions are answered.

\section{Assumptions and Exceptions} \label{AssumptAndExcept}
ComplexT:

\begin{enumerate}

\item I assumed that the real and imaginary parts would be represented in float numbers.  

\item For get\_phi, since The phase of a complex number with both real and imaginary parts as 0 is undefined, I returned None if both real and imaginary parts of the complex are 0

\item Since the reciprocal of 0 is undefined, I assumed the reciprocal of 0 complex number to be None in recip

\item since any number when divided by 0 is undefined, I assumed the quotient of a division by 0 to be None in div

\end{enumerate}

\noindent TraingleT:

\begin{enumerate}

\item I assumed that the lengths of all three sides of the triangle would be represented by integers

\item I assumed that a triangle is invalid if any of the sides is less than or equal to 0 in is\_valid

\item I assumed that the triangle is invalid if the sum of any two sides is not greater than the third side in is\_valid

\end{enumerate}
\section{Test Cases and Rationale} \label{Testing}


\section{Results of Testing Partner's Code}


\section{Critique of Given Design Specification}


\section{Answers to Questions}

\begin{enumerate}[(a)]

\item 
\item ...

\end{enumerate}

\newpage

\lstset{language=Python, basicstyle=\tiny, breaklines=true, showspaces=false,
  showstringspaces=false, breakatwhitespace=true}
%\lstset{language=C,linewidth=.94\textwidth,xleftmargin=1.1cm}

\def\thesection{\Alph{section}}

\section{Code for complex\_adt.py}

\noindent \lstinputlisting{../src/complex_adt.py}

\newpage

\section{Code for triangle\_adt.py}

\noindent \lstinputlisting{../src/triangle_adt.py}

\newpage

\section{Code for test\_driver.py}

\noindent \lstinputlisting{../src/test_driver.py}

\newpage

\section{Code for Partner's complex\_adt.py}

\noindent \lstinputlisting{../partner/complex_adt.py}

\section{Code for Partner's triangle\_adt.py}

\noindent \lstinputlisting{../partner/triangle_adt.py}

\end {document}